\documentclass{article}
\usepackage{amsmath}
\usepackage{amssymb}
\usepackage{graphicx}
\usepackage{hyperref}
\usepackage{xcolor}
\usepackage{booktabs}

\title{Mathematical Formulations in PAMR Routing}
\author{PAMR Project}
\date{\today}

\begin{document}

\maketitle

\section{Introduction}
This document provides a comprehensive collection of the mathematical equations used in the Pheromone-based Adaptive Multi-path Routing (PAMR) project and its comparisons with traditional routing protocols like OSPF and RIP.

\section{PAMR Core Routing Equations}

\subsection{Path Quality Calculation}
The comprehensive path quality calculation in PAMR considers multiple factors with weighted contributions:

\begin{align}
\text{delay\_quality} &= \frac{10.0}{\text{total\_delay} + 1.0} \\
\text{congestion\_impact} &= 0.5 \times \text{avg\_congestion} + 0.5 \times \text{max\_congestion} \\
\text{congestion\_quality} &= \frac{1.0}{1.0 + \text{congestion\_impact} \times 3.0} \\
\text{bandwidth\_quality} &= \frac{\text{min\_bandwidth}}{10.0} \\
\text{hop\_quality} &= \frac{1.0}{1.0 + \text{hop\_count} \times 0.1} \\
\text{final\_quality} &= \text{delay\_quality} \times 0.4 + \text{congestion\_quality} \times 0.3 + \text{bandwidth\_quality} \times 0.2 + \text{hop\_quality} \times 0.1
\end{align}

\subsection{Congestion-Induced Delay}
PAMR calculates how congestion affects delay using the following equation:
\begin{align}
\text{congestion\_delay} = \text{distance} \times (1 + \text{congestion}^2 \times 5)
\end{align}

\subsection{Edge Weight Calculation for Path Finding}
The core path finding algorithm uses edge weights based on:
\begin{align}
\text{congestion\_factor} &= 1 + \text{congestion}^2 \times 10 \\
\text{pheromone\_factor} &= \frac{1}{\text{pheromone} + 0.1} \\
\text{edge\_weight} &= \text{edge\_distance} \times \text{congestion\_factor} \times \text{pheromone\_factor} \times 0.1
\end{align}

\subsection{Alternative Path Weight Functions}
For finding alternative paths with higher congestion penalty:
\begin{align}
\text{congestion\_factor} &= 1 + \text{congestion}^3 \times 15 \\
\text{weight} &= \text{distance} \times \text{congestion\_factor}
\end{align}

For finding paths that prioritize low congestion:
\begin{align}
\text{congestion\_factor} &= (1 + \text{congestion} \times 10)^4 \\
\text{weight} &= \text{distance} \times \text{congestion\_factor}
\end{align}

\subsection{Node Selection Probability Components}
\begin{align}
\text{pheromone\_factor} &= \text{pheromone}^{\alpha} \\
\text{distance\_factor} &= \left(\frac{1.0}{\text{distance}}\right)^{\beta}
\end{align}

\subsection{Progressive Congestion Avoidance}
PAMR implements a tiered sensitivity to congestion levels:
\begin{align}
\text{congestion\_factor} = 
\begin{cases}
(1.0 - \text{predicted\_congestion})^{\gamma} & \text{if } \text{predicted\_congestion} < 0.3 \\
(1.0 - \text{predicted\_congestion})^{\gamma \times 2} & \text{if } \text{predicted\_congestion} < 0.6 \\
(1.0 - \text{predicted\_congestion})^{\gamma \times 3} & \text{otherwise}
\end{cases}
\end{align}

\subsection{Combined Desirability for Node Selection}
\begin{align}
\text{desirability} = \text{pheromone\_factor} \times \text{distance\_factor} \times \text{congestion\_factor} \times \text{bandwidth\_factor} \times \text{heuristic\_factor}
\end{align}

\section{OSPF Routing Equations}

\subsection{OSPF Link Cost}
\begin{align}
\text{cost} = \frac{\text{reference\_bandwidth}}{\text{link\_bandwidth}}
\end{align}
where reference\_bandwidth is typically 100,000 Mbps.

\subsection{Path Quality Equation (OSPF)}
\begin{align}
\text{path\_quality} = \frac{1.0}{\text{total\_distance} \times (1 + \text{max\_congestion})}
\end{align}

\section{RIP Routing Equations}

\subsection{RIP Metric Calculation}
\begin{align}
\text{metric} = \min(15, \max(1, \text{int}(\text{reference\_bandwidth} / \text{bandwidth})))
\end{align}
where reference\_bandwidth is typically 10 Mbps for RIPv2.

\subsection{Path Quality Equation (RIP)}
For fair comparison with PAMR, the same quality metric is used:
\begin{align}
\text{path\_quality} = \frac{1.0}{\text{total\_distance} \times (1 + \text{max\_congestion})}
\end{align}

\section{Common Network Metrics}

\subsection{Congestion Calculation}
\begin{align}
\text{congestion} = \min(0.95, \text{traffic} / \text{capacity})
\end{align}

\end{document}